% Appendix Template

\chapter{Enseignements et encadrements} % Main appendix title

\label{Annexes3} % Change X to a consecutive letter; for referencing this appendix elsewhere, use \ref{AppendixX}

\lhead{Annexes 3. \emph{Enseignements}} % Change X to a consecutive letter; this is for the header on each page - perhaps a shortened title\section{Encadrements de stagiaires}

\section{Encadrements de stagiaires}

Anaïs \textsc{Chanson}. 2015. Mise en évidence d'une reconnaissance interspécifique du signal larvaire chez deux espèces de Diptères. Master 1\up{ère} année, Université de St Etienne.

Aurore \textsc{Becquart}. 2015. Etude de l'activité locomotrice des larves nécrophages face à des signaux olfactifs pertinents. Master 1\up{ère} année, Université Paris XIII.

Gwennan \textsc{Giraud}. 2014. Etude du comportement des larves nécrophages (\textit{Lucilia sericata}) dans des tests de choix en olfactomètre. Master 1\up{ère} année, Université de Rennes 1.

Cécile \textsc{Betremieux}. 2013. Etude comportementale de l'effet dose du signal larvaire (\textit{Lucilia sericata}) et du scanning, un comportement exploratoire caractéristique. Master 1\up{ère} année, Université de Paris XIII.

Fanny \textsc{Catteau}. 2013. Etude d'un possible effet attractif/rétentif du cholestérol sur les larves nécrophages de Diptères. Licence 2\up{ème} année, Université Catholique de Lille.

Pierre \textsc{Hainselin}. 2012. Etude du comportement des larves nécrophages (\textit{Lucilia sericata}) dans des tests de choix binaires. Licence 3\up{ème} année, Université Catholique de Lille.

\section{Enseignements}

Intervention dans le Master 2 Taphonomie - Comportement des insectes nécrophages

Travaux pratiques en Licence 2 Biologie des Organismes - Zoologie
