% Chapter Template

\chapter*{Objectifs et plan de la thèse} % Main chapter title
\addcontentsline{toc}{chapter}{Objectifs et plan de la thèse}
\label{Methodo} % Change X to a consecutive number; for referencing this chapter elsewhere, use \ref{ChapterX}

\lhead{\emph{Objectifs et Méthodologie générale}} % Change X to a consecutive number; this is for the header on each page - perhaps a shortened title

\textit{``Comme symbole d'effronterie et d'impertinence, il faudrait prendre la mouche. Tandis que tous les animaux, en effet, craignent l'homme au-dessus de tout et le fuient déjà de loin, la mouche, elle, se pose sur son nez."}

\begin{flushright}
Arthur Schopenhauer (1788-1860)
\end{flushright}

\cleardoublepage

%----------------------------------------------------------------------------------------
%	SECTION 1
%----------------------------------------------------------------------------------------

	\section{Objectifs}
Cette thèse a pour objectifs de mettre en évidence et de quantifier les dynamiques d’agrégation des larves nécrophages de Diptères et les mécanismes qui sous tendent ces regroupements. Elle s'inscrit dans deux contextes bien distincts : un fondamental qui est l'étude des comportements collectifs et le second, que l'on peut qualifier d'applicatif, qui est l'entomologie forensique. Bien que le côté théorique soit le plus présent dans cette thèse, l'éventuel apport applicatif de ce travail en contexte judiciaire ne doit pas être négligé.

Comme présentée précédemment dans l'Introduction générale, l'étude de l'agrégation existe chez de nombreux taxons animaux, et en particulier chez les insectes. A l'heure actuelle, aucune étude de ce type n'a cependant été réalisée sur les larves nécrophages de Diptères. Ces insectes ont un comportement d'agrégation très marqué, ayant pour conséquence une forte modification de leur environnement (e.g. génération de chaleur ou production d'enzymes). Les agrégats formés sont, sur un même cadavre, composés de plusieurs espèces.

Les Diptères (e.g. les mouches) sont, à l'heure actuelle, les principaux insectes étudiés et utilisés pour la datation du décès \cite{charabidze_insectes_2014}. Les durées de développement de ces insectes ont été largement étudiées, permettant ainsi d'affiner les techniques de datation utilisées par les experts. En revanche, le comportement des insectes reste mal connu et peu étudié. Cette thèse a pour objectif d'étudier le comportement des Diptères, et plus spécifiquement celui de leurs larves. Ce travail offre une vision comportementale inédite de ces espèces d'intérêt forensique, qui s'ajoutera à une vision physiologique déjà bien établie.

De part ces observations \textit{in natura}, les larves nécrophages sont un modèle d'étude inédit et pertinent dans ce double contexte d'étude (i.e. comportements collectifs et forensique).




%----------------------------------------------------------------------------------------
%	SECTION 2
%----------------------------------------------------------------------------------------

    \section{Plan}

Cette thèse comporte cinq chapitres correspondant chacun à un article publié (cf. Chapitres \ref{Chapter2}, \ref{Chapter3}, \ref{Chapter4}) ou soumis (cf. Chapitres \ref{Chapter1}, \ref{Chapter5}) dans une revue internationale à comité de lecture.

Le premier chapitre est une revue de la littérature (ARTICLE 1) sur les groupes hétérospécifiques chez les arthropodes. Cette revue recense les articles où les auteurs ont observé, décrit ou étudié des groupes hétérospécifiques d'arthropodes dans le contexte de l'étude des comportements collectifs. L'écriture de cette revue fait suite à la constatation de l'absence d'une telle synthèse chez ce groupe animal. L'existence de tels groupes dans la nature remet quelque peu en cause la définition commune de \textit{l'agrégation}. Ils posent également des questions sur la frontière entre les phénomènes de coopération et de compétition entre les espèces. Ces groupes nous offrent un angle d'étude particulièrement intéressant sur ces phénomènes et notamment sur le processus de spéciation sympatrique.

\begin{itemize} 
\item[\tiny{$\blacksquare$}] ARTICLE 1 - Mixed-species aggregation in arthropods - a review\\ (\textbf{soumis} à \textit{Behavioral Ecology}).
\end{itemize}

Le second chapitre (ARTICLE 2) traite de la mise en évidence du comportement actif d'agrégation des larves de \textit{Lucilia sericata} (Diptera: Calliphoridae) sur un milieu nutritif homogène. Cet article démontre l'existence d'un signal déposé au sol par les larves et reconnu par leurs congénères.

\begin{itemize} 
\item[\tiny{$\blacksquare$}] ARTICLE 2 - Evidence of active aggregation behaviour in \textit{Lucilia sericata} larvae and possible implication of a conspecific mark (\textbf{publié} dans \textit{Animal Behaviour}).
\end{itemize}


Le troisième chapitre (ARTICLE 3) s'attache à décrire et quantifier un comportement exploratoire caractéristique des larves nécrophages : le \textit{scanning}. Ce comportement avait été jusqu'alors associé à la recherche de nourriture. Cette étude montre que le scanning est également impliqué dans la recherche de congénères. De plus, ce travail met en évidence les capacités de suivi de trace des larves de \textit{L. sericata}.

\begin{itemize} 
\item[\tiny{$\blacksquare$}] ARTICLE 3 - A first insight in the scanning behaviour of presocial blow fly larvae\\ (\textbf{publié} dans \textit{Physiological Entomology}).
\end{itemize}

Le quatrième chapitre (ARTICLE 4) met en évidence la prise de décision collective chez deux espèces de larves, \textit{L. sericata} et \textit{Calliphora vomitoria}. Ces espèces sont capables de choisir collectivement un site d'agrégation et ce, qu'elles appartiennent à un groupe monospécifique ou hétérospécifique. Cette étude est la première, à notre connaissance, à quantifier la dynamique d'agrégation d'un groupe hétérospécifique d'insectes.

\begin{itemize} 
\item[\tiny{$\blacksquare$}] ARTICLE 4 - Interspecific shared collective decision-making in two forensically important species (\textbf{resoumis} à \textit{Proceedings of the Royal Society of London B}).
\end{itemize}

Le cinquième chapitre (ARTICLE 5) démontre le choix collectif des larves pour une température. Nous avons démontré l'existence de préférendum thermique spécifique chez les larves nécrophages. Sur un gradient de température, les larves sélectionnent collectivement une température selon l'espèce à laquelle elles appartiennent.

\begin{itemize} 
\item[\tiny{$\blacksquare$}] ARTICLE 5 - Thermoregulation in gregarious Dipteran larvae: evidence of species-specific temperature selection (\textbf{soumis} à \textit{Entomologia Experimentalis et Applicata}).
\end{itemize}

Enfin, ce travail s'achève avec une Discussion, qui s'attache à replacer ces travaux dans un contexte évolutif plus large. Elle apporte des perspectives de travail sur l'agrégation des larves nécrophages et la portée plus générale de ces études.

Dans une envie de clarté et de continuité, les références bibliographiques de ce travail (format numéroté) ont toutes été listées en fin de manuscrit (cf. Bibliographie générale). Ce choix assigne à une référence bibliographique un numéro unique pour tout le document. 

\clearpage


    