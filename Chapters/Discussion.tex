% Chapter Template
\setcounter{chapter}{6}
\chapter*{Discussion et conclusion} % Main chapter title
\addcontentsline{toc}{chapter}{Discussion et conclusion}
\label{Discussion} % Change X to a consecutive number; for referencing this chapter elsewhere, use \ref{ChapterX}

\lhead{\emph{Discussion et conclusion}} % Change X to a consecutive number; this is for the header on each page - perhaps a shortened title

\textit{``L'égalité, la seule égalité en ce monde, l'égalité devant l'asticot."}

\begin{flushright}
Jean-Henri Fabre (1823-1915)
\end{flushright}

\cleardoublepage

%----------------------------------------------------------------------------------------
%	SECTION 1
%----------------------------------------------------------------------------------------
    
	\section{Les larves nécrophages: un nouveau modèle d'étude}  
    
Les larves nécrophages de Diptères étaient jusqu'à présent étudiées principalement en entomologie forensique, un cadre scientifique à visée applicative immédiate. En effet, ces insectes sont d'importance en entomologie forensique car ils arrivent très tôt sur un cadavre et sont donc utilisés par les experts pour effectuer une datation du décès (calcul de l'Intervalle Post-Mortem minimum (IPMmin); cf. Introduction \ref{subsubsec:forensique}). Des études physiologiques sur le développement des larves ont été nécessaires pour affiner ces techniques de datation. Ces études ont permis, entre autres, de connaître les durées de développement en fonction de la température \cite{grassberger_effect_2001}, l'impact de la qualité nutritive d'une ressource sur les larves \cite{ireland_effects_2006} ou encore les effets des produits chimiques sur leur développement \cite{aubernon_experimental_2015}. La physiologie des larves est donc bien connue, mais leur comportement reste encore peu étudié, comme le souligne \citet{rivers_physiological_2011} dans leur revue.

Notre approche éthologique amène une vision nouvelle de la vie de ces insectes, et  aborde notamment des questions fondamentales. L'étude des comportements collectifs des larves nécrophages s'ajoute à celle faite sur d'autres espèces sociales comme les blattes \cite{canonge_individuel_2011}, les fourmis \cite{dussutour_organisation_2004}, les poissons \cite{robert_does_2013} ou encore les oiseaux \cite{farine_collective_2014}. Les larves nécrophages sont un modèle biologique inédit offrant des perspectives intéressantes dans ce cadre d'étude. En effet, l'observation quasi systématique de groupes hétérospécifiques et la capacité des larves à modifier leur environnement de manière significative permettent l'étude de plusieurs phénomènes comportementaux et évolutifs. Parmi ces phénomènes, on peut notamment citer les phénomènes de coopération-compétition qui peuvent exister entre deux espèces appartenant à un même groupe social (cf. Chapitre \ref{Chapter1}).     



%----------------------------------------------------------------------------------------
%	SECTION 2
%----------------------------------------------------------------------------------------     
		\section{L'individu}     
Nos études réalisées sur le comportement individuel offrent un aperçu des mécanismes qui sous-tendent la formation et le maintien des groupes de larves. L'un de ces mécanismes étant le signal déposé au sol passivement par les larves et reconnu par les congénères. Ce type de marquage chimique tactile a été décrit chez plusieurs espèces d'insectes comme les fourmis \cite{devigne_how_2006} ou encore les coccinelles \cite{durieux_role_2012}. Les molécules responsables de cette reconnaissance sont généralement des hydrocarbures cuticulaires. 

			\subsection{Attraction/rétention du signal larvaire}
Nos tests de choix binaire ont mis en évidence le dépôt au sol passif d'un composé chimique (cf. Chapitre \ref{Chapter2}). Ce dépôt a un effet d'attraction et de rétention sur les congénères \cite{boulay_evidence_2013}. De plus, les larves sont capables, via le scanning, de détecter ce signal et de le suivre (cf. Chapitre \ref{Chapter3}). Ces observations suggèrent que ce signal larvaire serait impliqué dans l'agrégation de ces espèces. Cette attraction à très courte distance nécessite d'être testée à plus grande distance à l'aide d'olfactomètres, comme ce qui a été fait chez le lombric \cite{zirbes_earthworms_2011}. \citet{zirbes_earthworms_2011} ont observé une capacité des lombrics à détecter à distance et à rejoindre les signaux volatils émis par des congénères.

Des tests complémentaires de choix binaires (cf. Annexe \ref{Annexes1}, expérience 2 du Chapitre \ref{Chapter2}) ont montré que des larves de \textit{Calliphora vomitoria} passent significativement plus du temps dans une zone où 10 larves de \textit{Lucilia sericata} avaient été laissées pendant 10min (comparativement à une zone vierge de dépôt). Ce résultat suggère une reconnaissance interspécifique du signal ainsi que la conservation de son effet attractif/rétentif chez une espèce proche. Une telle reconnaissance est nécessaire pour l'établissement et le maintien des groupes contenant plusieurs espèces (cf. Chapitre \ref{Chapter1}). La proximité anatomique des organes  sensorielles des larves entre les espèces (cf. Introduction \ref{subsubsec:anatomie}, \citep{chu-wang_fine_1971}) et l'observation d'un même comportement exploratoire caractéristique (scanning) laissent à penser que la reconnaissance du signal larvaire seraient sous tendue par les mêmes mécanismes physiologiques et comportementaux, favorisant ainsi l'existence de vecteurs d'agrégation interspécifiques.


			\subsection{Identification du marquage chimique}
Une première analyse chromatographique du signal larvaire a été entreprise chez \textit{Lucilia sericata} avec l'équipe du Pr. Georges \textsc{Lognay} (Départ. Chimie Analytique, Agro-Bio-Tech Gembloux). Les résultats ont montré une composition largement dominée par le cholestérol (Cholest-5-en-3-ol(3$\beta$); cf. Annexe \ref{Annexes1}). Les tests comportementaux préliminaires n'ont pas mis en évidence d'attraction/rétention significative des larves pour ce composé (protocole basé sur l'expérience 2 du Chapitre \ref{Chapter2}). Des tests plus approfondis seraient nécessaires pour identifier précisément chaque composé chimique et les tester sur les individus. Des études d'identification des hydrocarbures cuticulaires ont déjà été réalisées chez les larves nécrophages de Diptères \citep{roux_ontogenetic_2008,zhu_development_2006}, mais, à l'heure actuelle, la composition chimique du dépôt laissé par les individus n'est pas connue.

Dans un premier temps les composés lourds seraient identifiés via une chromatographie en phase gazeuse. Puis dans une seconde étape, les composés volatils seraient analysés suivant le protocole de \citet{frederickx_volatile_2012}. En effet, si l'effet rétentif du signal est vraisemblablement porté par des molécules lourdes (i.e. longues chaînes carbonnées qui restent au sol), nous pouvons penser que l'effet attractif serait supporté par les molécules volatiles. Des tests comportementaux d'attraction et de rétention avec les composés identifiés permettraient ainsi de mettre en évidence les molécules impliquées.

D'après nos observations en individuel et en collectif d'une reconnaissance interspécifique des vecteurs d'agrégation (cf. Chapitre \ref{Chapter4}), une analyse comparative des profils chromatographiques des signaux déposés par \textit{L. sericata} et \textit{C. vomitoria} est une approche à privilégier. En effet, elle permettrait d'identifier les composés en communs et de les tester en priorité sur les individus des deux espèces dans des tests de choix.
            

%----------------------------------------------------------------------------------------
%	SECTION 3
%----------------------------------------------------------------------------------------       

		\section{Le groupe}
Le comportement d'agrégation des larves semble être une réponse adaptative efficace face aux contraintes liées à cet environnement particulier qu'est le cadavre. En effet, un cadavre est un milieu instable, à haute valeur nutritionnelle donc compétitif mais éphémère. A l'heure actuelle, deux bénéfices de l'agrégation sont bien connus et étudiés dans la littérature : la génération de chaleur \citep{slone_thermoregulation_2007,charabidze_larval-mass_2011} et la production accrue par le nombre de liquides d'excrétion/sécrétion riches en enzymes et en antibiotiques \citep{wilson_impacts_2015,sandeman_tryptic_1990}. \citet{rivers_physiological_2011} avancent d'autres bénéfices liés à l'agrégation, notamment une protection contre les prédateurs, une limitation de la dessiccation et une meilleure assimilation de la nourriture. Bien que ces avantages soient, au regard d'autres études sur les agrégations d'arthropodes \cite{jeanson_key_2012}, intuitifs, ils n'ont pas été mis en évidence expérimentalement. Tout comme ces bénéfices, les mécanismes qui sous-tendent la stabilisation des masses larvaires sont méconnus et peu étudiés.

Nous avons vu que le comportement d'agrégation des larves est très marqué et que la formation des groupes est rapide (cf. Chapitres \ref{Chapter2} et \ref{Chapter4}). Nos résultats ont démontré que l'agrégation des larves était active et n'était pas seulement la résultante de l'agrégation des pontes par les femelles \cite{fenton_oviposition_1999}. Dans nos expériences, un comportement de thigmotactisme positif a été observé (cf. Chapitre \ref{Chapter2}), comportement également présent chez d'autres espèces d'arthropodes (e.g. cloportes \cite{devigne_individual_2011} ou acariens \cite{mailleux_collective_2011}). Ce comportement peut expliquer la localisation proche des parois de l'arène de l'agrégat observé en début d'expérience (cf. Chapitre \ref{Chapter2}). En revanche, après 24h, l'agrégat s'était déplacé, suggérant un thigmotactisme interindividuel plus important dû fait d'un nombre d'individus accru présents au sein du groupe. Ce mouvement régit par le contact entre individus est également observé chez les criquets grégaires \cite{simpson_gregarious_2001}. Ces insectes forment d'immenses agrégats pouvant contenir des milliers d'individus ravageant les cultures \citep{buhl_using_2012,buhl_group_2010}. Ces bancs de criquets peuvent prendre différent patterns : l'espèce \textit{Chortoicetes terminifera} (présente en Australie) forme un groupe avec un front dense alors que l'espèce \textit{Locustana pardalina} (présente dans le sud de l'Afrique) est observée en bancs formés par des colonnes d'individus. Les mécanismes aboutissant à ces patterns restent méconnus (e.g. impact de la végétation, comportement individuel) mais la démarche de ces auteurs alliant modélisation mathématique et observations en laboratoire/terrain est intéressante. Cette démarche offre une piste de recherche supplémentaire pour l'étude du comportement d'agrégation des larves nécrophages.

		\subsection{Dynamique d'agrégation}
Les dynamiques de choix observées avec des groupes monospécifiques (cf. Chapitre \ref{Chapter4}) sont comparables à celles retrouvées chez les blattes \citep{canonge_self-amplification_2009,jeanson_self-organized_2005} ou les cloportes \cite{devigne_individual_2011}. Un choix clair pour un site d'agrégation est fait en quelques minutes par les larves (en moins de 5min, 50$\%$ des individus ont sélectionné un spot) et ce choix est maintenu dans le temps. Cette prise de décision collective peut être définie comme étant une décision consensuelle selon \citet{conradt_consensus_2005}. En effet, aucun conflit intérêt\footnotemark[1]\footnotetext[1]{\textit{Conflit d'intérêt}: apparaît quand un individu ou une organisation est impliquée dans de multiples intérêts, l'un d'eux pouvant corrompre la motivation à agir sur les autres} n'est observé et au vu des organes sensoriels des larves et de leur sensitivité (cf. Introduction \ref{subsubsec:anatomie}), une communication au niveau local\footnotemark[2]\footnotetext[2]{\textit{Communication locale}: les membres d'un groupe ne peuvent communiquer qu'avec leur plus proche voisin \cite{conradt_consensus_2005}.} est à privilégier. Ces caractéristiques aboutissent, selon \citet{conradt_consensus_2005}, à une décision collective classée de consensuelle. Les mécanismes régissant ce type de décision sont ceux de l'auto-organisation (cf. Introduction \ref{sec:autoorganisation}). Des études comportementales restent cependant à réaliser pour étayer l'idée d'un système auto-organisé. Par exemple, l'existence d'un éventuel quorum\footnotemark[3]\footnotetext[3]{\textit{Quorum}: nombre minimum d'individus nécessaire pour prendre ou favoriser une action particulière influençant l'ensemble du groupe à adopter cette action.} pourrait être analysée en faisant varier le nombre d'individus placés dans l'arène (e.g. 10, 20 ou 30 individus). Les études sur les blattes sont un bon exemple de ce phénomène et un modèle de comparaison intéressant \citep{sempo_complex_2009,jeanson_self-organized_2005}.

Notre travail a mis en évidence la dynamique d'agrégation d'un groupe composé de deux espèces (cf. Chapitre \ref{Chapter4}). Un choix collectif pour un site est observé, mais ce choix se fait plus lentement qu'avec un groupe composé par une seule espèce (cf. Chapitre \ref{Chapter4}). Ces observations laissent sous entendre que les deux espèces de larves partagent les mêmes mécanismes d'agrégation, ou du moins une partie. Une reconnaissance interspécifique du signal déposé par les larves est notamment suggérée  (cf. Chapitre \ref{Chapter2}). Cette reconnaissance interspécifique serait moins efficace qu'une reconnaissance intraspécifique, créant une dynamique d'agrégation du groupe plus lente (comme observée). Cela laisse également à penser que ces 2 espèces pourraient, en conditions naturelles, coopérer pour aboutir à des tailles de groupe suffisamment importante, probablement dans le but ultime d'obtenir les bénéfices liés à l'agrégation (effet Allee). Des études similaires réalisées en faisant varier la quantité de ressources et le nombre d'individus de chaque espèce permettraient d'étudier une éventuelle limite entre coopération et compétition. De plus, ce type d'études offrirait une vision des phénomènes d'exclusion/ségrégation entre les espèces observées dans la nature (cf. Chapitre \ref{Chapter1}). 

Une telle ségrégation peut être impactée par la température locale. Nous avons observé que chaque espèce avait un préférendum thermique (cf. Chapitre \ref{Chapter5}). Comme prochaine étape, il serait intéressant d'observer deux populations de larves d'espèces différentes placées sur le gradient thermique utilisé dans l'étude du Chapitre \ref{Chapter5} (Thermograde). Ce type d'expérience, actuellement en cours au laboratoire (thèse de C. \textsc{Aubernon}), permettra d'étudier la place du comportement social face aux préférences thermiques des larves. Est-ce qu'une ségrégation du groupe va être observée avec chaque espèce présente au niveau de sa température préférentielle ? Ou à l'inverse un groupe mixte formé à une température intermédiaire ? Si une ségrégation est observée, la recherche d'une température préférée passerait en priorité sur la formation d'un groupe mixte. La température serait un vecteur d'agrégation lié à l'espèce. \textit{A contrario}, si un groupe interspécifique est formé cela supposerait que les larves cherchent dans un premier temps à s'agréger avec des congénères même si la température recherchée est à proximité. Cette dernière observation supposerait que les vecteurs d'agrégations interspécifiques, autre que la chaleur, seraient prédominants sur la température. Ce compromis entre agrégation et température est une piste de recherche intéressante et novatrice qui permettra de mieux comprendre les avantages recherchés par les larves.

		\subsection{Attraction/rétention du groupe}
Les phénomènes d'attraction/rétention que peut avoir un groupe sur un individu sont à la base de l'émergence des systèmes complexes. Ils sont observés chez les blattes \citep{canonge_individuel_2011,lihoreau_collective_2010}, les cloportes \cite{devigne_individual_2011} ou encore les chenilles \cite{fitzgerald_specificity_1979}. Ces phénomènes conduisent à des amplifications du système (cf. Introduction \ref{sec:amplification}), ces mécanismes étant liés aux structures auto-organisées. 

L'effet attractif permet à la larve de localiser, plus ou moins à distance, le groupe. Dans notre système larves/sites de nourriture (cf. Chapitre \ref{Chapter4}), le suivi d'un individu met en évidence à la fois une attraction et une rétention du groupe. Des tests préliminaires ont montré qu'une larve de \textit{Lucilia sericata} était également attirée par un groupe de congénères présent à distance. Dans un olfactomètre à deux branches avec pour choix un groupe de 40 individus et un témoin (vide), les larves ont choisi dans 100$\%$ des essais la branche contenant le groupe (cf. Annexe \ref{Annexes1}). Ce résultat préliminaire sous-tend une attraction à distance du groupe sur l'individu. Ce type d'effet a été mis en évidence chez les groupes de blattes \cite{ame_cockroach_2004} ou encore les lombrics \cite{zirbes_earthworms_2011}.

L'effet rétentif va maintenir la cohésion sociale entre les membres du groupe. Basés sur le modèle blatte \cite{canonge_individuel_2011}, des tests comportementaux où la taille du groupe va varier vont nous permettre de complémenter notre observation d'un tel effet chez les groupes de larves (cf. Chapitre \ref{Chapter4}). Associés à l'effet attractif du groupe (recherche de congénères), ces effets rétentifs amplifient le système et aboutissent à la formation des structures auto-organisées. Il convient, cependant, de garder à l'esprit que des phénomènes d'encombrement peuvent être observés au sein des agrégats, se traduisant par une bousculade permanente pour l'accès à la nourriture (\textit{scramble competition} \cite{rivers_physiological_2011}).

%----------------------------------------------------------------------------------------
%	SECTION 4
%----------------------------------------------------------------------------------------       
		\section{La modélisation comme prochaine étape}  
        
Notre étude est la première étape vers une compréhension plus globale des groupes interspécifiques. En effet, dans leur ouvrage de référence, \citet{camazine_self-organization_2001} proposent une méthodologie à adopter lors de l'étude des comportements collectifs (appelée \textit{bottom-up}; Figure \ref{fig:model}). Ils présentent 4 étapes : (i) la description du phénomène collectif et notamment sa dynamique, (ii) l'identification des comportements individuels et des interactions interindividuelles qui permettent d'expliquer la dynamique collective, (iii) la formulation d'un modèle sur la base des règles comportementales caractérisées précédemment, et enfin (iv) la comparaison entre la dynamique collective obtenue avec le modèle et celle acquise expérimentalement. Cette démarche montre bien le va-et-vient nécessaire entre les observations expérimentales et la modélisation mathématique. Ce cycle de travail amène à une meilleure compréhension du système collectif étudié.

La modélisation mathématique a ceci d'avantageux qu'elle permet de caractériser l'organisation et la structure d'un groupe composé d'une multitude d'individus avec un nombre d'événements (e.g. interactions sociales) important. Elle offre plusieurs avantages à l'utilisateur une fois celle-ci fidèle aux observations expérimentales. Elle permet d'avoir un nombre de réplications bien plus important qu'en conditions expérimentales et de faire varier les paramètres aisément (e.g. taille du groupe ou nombre d'abris). Outre ces avantages, la modélisation mathématique, couplée à la simulation, permet de déterminer les vecteurs nécessaires et suffisants pour produire les phénomènes observés. Cette démarche simplifie la compréhension des systèmes complexes. Des premières réflexions sur cette étape ont été entamées avec G. \textsc{Sempo} et A. \textsc{Campo} (Unité d'Écologie Sociale, ULB), et offriront une base de travail pour une étude de modélisation.

\begin{figure}[ht]
	\centering
		\includegraphics[width=0.8 \textwidth]{Figures/modelisation.png}
		\rule{35em}{0.5pt}
	\caption[Model]{Schéma présentant les différentes étapes lors de l'étude des comportements collectifs (tiré de \citet{camazine_self-organization_2001}).}
	\label{fig:model}
\end{figure}


%----------------------------------------------------------------------------------------
%	SECTION 5
%----------------------------------------------------------------------------------------     
    \section{Perspectives en entomologie forensique}
    
Ce travail de thèse a mis en évidence l'aspect fondamental de la vie sociale chez les larves nécrophages de Diptères, des espèces d'importance en entomologie forensique. Ces groupes sociaux impactent de manière significative le développement des larves. Lors d'une expertise judiciaire, il est primordial pour l'expert entomologiste de connaître les conditions dans lesquelles se sont développées les insectes retrouvés sur un corps (e.g. température, présence de substances toxiques, etc.). Ces conditions de vie vont directement affecter les durées de développement des insectes et donc la datation du décès. A l'heure actuelle, l'impact des conditions locales et du comportement larvaire ne sont que trop rarement pris en compte.

Il est facile d'imaginer qu'une larve de Diptères prélevée sur un cadavre au sein d'un groupe de quelques individus n'aura pas eu accès aux mêmes bénéfices liés au groupe qu'une larve extraite au milieu de milliers. Cette dernière aura bénéficié, entre autres, du dégagement de chaleur (i.e. génération significative observée avec des groupes au-delà de 1000 individus \cite{johnson_effect_2014}) accélérant son développement. Il est donc nécessaire de mettre en place des outils, qualitatifs dans un premier temps, permettant d'intégrer la variabilité due au comportement social des larves nécrophages lors d'une découverte de cadavre. Une fois nos connaissances approfondies sur l'impact de la taille du groupe sur le développement des larves, des outils quantitatifs viendront s'ajouter. Ces outils permettront, \textit{in fine}, d'affiner les techniques actuelles de datation entomologique du décès en tenant compte du comportement social des larves nécrophages.


%----------------------------------------------------------------------------------------
%	SECTION 6
%----------------------------------------------------------------------------------------         
    \section{Conclusion}    
    
Pour conclure, ce travail a mis en évidence les dynamiques de formation des groupes et les mécanismes qui sous-tendent ces regroupements des larves nécrophages de Diptères. Ce grégarisme impacte de manière significative la vie de ces insectes. La démarche expérimentale de va-et-vient entre le comportement individuel et collectif des larves adoptée dans ce travail ouvre à des études éthologiques et de modélisation futures. 

Un agrégat de plusieurs milliers de larves sur un cadavre peut être vu comme un \textit{immense estomac} (une idée avancée par D. \textsc{Charabidzé}). En effet, un groupe de larves nécrophages va produire des enzymes \cite{wilson_impacts_2015}, générer de la chaleur \cite{charabidze_larval-mass_2011}, modifier localement le pH et les populations bactériennes, et avoir une action mécanique. Ces différents éléments sont également retrouvés dans nos estomacs lors de la digestion. En s’agrégeant en immenses groupes interspécifiques, les larves recréent donc des conditions propices à la liquéfaction des chairs et à leur assimilation, favorisant ainsi le développement de chaque individu et l'exploitation rapide d'une ressource complexe. Si on se réfère à la définition de l'auto-organisation de Sumpter \cite{sumpter_collective_2009} : \textit{Le principe central de l'auto-organisation est que les interactions simples répétées entre les individus peuvent produire des modèles adaptatifs complexes au niveau du groupe}, le modèle adaptatif complexe d'un immense estomac semble pertinent. Avec l'apport de la modélisation mathématique et de la simulation, cette théorie laisse entrevoir des avancées significatives et inédites sur notre compréhension actuelle des comportements collectifs comme stratégies adaptatives.

\clearpage


        
        
        